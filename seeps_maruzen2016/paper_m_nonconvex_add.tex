\documentclass[11pt,a4paper]{article}

\usepackage{setspace}
\onehalfspacing
\usepackage[top=1.75in, left=1.55in, bottom=1.75in, right=1.55in]{geometry}

\begin{document}

\noindent
(以下を末尾に追加)\ \\
\ \\
非凸性や履歴効果は,主に生態学の分野で議論されてきたものであったが,
近年では経済モデルにも積極的に取り入れられるようになっている(de Zeeuw, 2014).
上記の浅い湖に限らず,
海洋生態系や,漁業資源,気候システム等,
経済活動に伴う環境負荷によって非連続的な変化が生じ得るシステムは多い.
予防原則の観点からは,
望ましくないレジームシフトがシステムに生じることのないよう
環境負荷を抑制することが求められるが,
一般には,費用と便益とを慎重に比較した上で
最適な環境・資源管理政策を考える必要がある.(阪本浩章)

\ \\
\ \\
\noindent
(以下を引用文献に追加)\ \\
\ \\
de Zeeuw, A. (2014) ``Regime shifts in resource management,''
\textit{Annual Review of Resource Economics}, 6, 85--104.

\end{document}