\documentclass[11pt,a4paper]{article}
\usepackage{setspace}
\onehalfspacing
\usepackage[top=1.75in, left=1.55in, bottom=1.75in, right=1.55in]{geometry}

\begin{document}

\noindent
世代間衡平性の公理的アプローチ\\
\ \\
\ \\
\ \\

経済学における\underline{公理的方法}とは,
問題の解やメカニズムを,
それが有し得る抽象的な性質(\underline{公理})の組合せによって特徴付ける方法を言う.
とくに\underline{社会的選択理論}の文脈では,効率性や衡平性のような,
社会的な意思決定のルールが満たすべき公理を列挙し,
それらを全て満たすルールが存在するか(つまり複数の公理が互いに整合的であるか),
もし存在するのであればそれはどのようなルールであるか,
あるいは逆に,特定のルールを所与としたときにそれがどのような公理の組を体現したものと言えるのか,
といったことが主要な関心事となる.

\noindent\textbf{●世代間衡平性の不可能性}\hspace{0.5em}
基本的な設定として,各世代が享受する効用の水準を,現在から無限期先の将来にかけて並べた経路を考える.
そのような経路は一般に複数存在し,
例えば将来世代の効用を犠牲にして
現在世代の効用を高める(あるいはその逆)といったことが技術的に可能であり得る.
したがって,規範的な立場からは,
複数の可能な選択肢の中から一本の経路を選び出すための評価基準が必要になる.
当然ながら,
選択される経路は評価基準に依存するから,
その基準にどのような性質を要求すべきか(どのような公理を満たすことを要求するか)ということが問題となる.
とくに地球温暖化問題のように,
利害関係者である将来世代との交渉が不可能であり,
実質的に現在世代だけで経路を選択しなければならないような問題では,
社会的な意思決定の基準は\underline{世代間衡平性}に配慮したものであることが望ましい.
さもなければ,現在世代は自らの選択を(それがどのようなものであれ)正当化することが困難になるからである.

社会的な意思決定の基準を考えるにあたって,
それが満たすべき公理としてしばしば挙げられるのは,
二つの異なる経路を比較した時に必ず優劣がつけられること(完備性),
三つ以上の経路について優劣の順序が循環しないこと(推移性),
無駄のある経路が望ましい経路として選ばれないこと(パレート原理),
いずれの世代も偏りなく扱うこと(匿名性),
などである.
いずれももっともらしい要請であるが,
一連の研究によって,これら全てを同時に満たす意思決定の基準は(明示的に定義できる形では)存在しないことが既に分かっている
(Diamond, 1965; Basu and Mitra, 2003; Lauwers, 2010).
これは言い替えれば,
いかなる社会的な意思決定の基準も,
上記の公理のいずれかに必ず反するということである.
したがって,合理的な基準に求められる基本的な要請(完備性や推移性)を満たそうとする限り,
効率性(パレート原理)と衡平性(匿名性)はトレードオフを免れないということになる.

\noindent\textbf{●世代間衡平性の可能性}\hspace{0.5em}
世代間衡平性を社会的な意思決定の中で考慮するためには,
上記の公理のいずれかについて,それを放棄するか,あるいは少なくとも何らかの形で修正しなければならない.
考えられる方法の1つは,
完備性や推移性といった,
意思決定の基準に求められる基本的な性能を部分的に放棄することである.
具体的には,効用和を評価基準とする\underline{功利主義}や,
最も不遇な世代の効用のみを評価基準とする\underline{マキシミン}(あるいは辞書的なマキシミンである\underline{レキシミン})について,%といった古典的な基準を
これらを無限期間に拡張したものがその例である.
いずれの基準も完備性を放棄し,
経路の裾(ある有限期先を出発点とする部分経路)がパレート原理で比較可能なものに比較対象を限定することで,
効率性と世代間衡平性を同時に満たすことを可能にしている(Basu and Mitra, 2007; Bossert et al., 2007).
一方で,推移性を若干弱めれば,
完備性を放棄することなくパレート原理と匿名性をかなり強い意味で維持できることも知られている(Sakai, 2010).

別の方向性として,
世代間衡平性や効率性の公理自体を弱めるということも考えられる.
例えばChichilnisky (1996)は,
世代間衡平性を匿名性によって捉えるのではなく,
遠い将来世代の厚生が無視されないこと(現在世代の非独裁)を要求する形で,
パレート原理と将来世代への配慮とが共存できる枠組みを提案している.
ただし,Chichilnisky (1996)の意思決定基準は定常性を満たさないため,
時間の経過とともに最適な経路が変化し得るという問題がある.
別のアイディアとして,
将来世代が現在世代に比べて不遇であるような場合にのみ衡平性を問題にする(将来世代に対するHammond衡平性)ことで,
効率性と衡平性との正面衝突を避けるという試みもある(Asheim et al., 2012).
逆に,効率性の要請を弱めることでも,衡平性との共存は可能になる.
例えば,パレート原理の適用対象をいずれの世代をとっても自分より不遇な世代が有限個しか存在しないような経路に限定すれば,
無限期間のレキシミン基準でも完備性を維持したり(Asheim and Zuber, 2012),
\underline{ランク割引功利主義}と呼ばれる基準で匿名性を実現したりすることも可能になる(Zuber and Asheim, 2012).

公理的アプローチによる分析は,
意思決定の基準を抽象的な公理によって特徴付けることで,
その基準を用いることが暗黙的に含意するものを明らかにする.
また,効率性や衡平性といったもっともらしい諸公理が実際には両立し得ないことを示すことで,
社会的な意思決定の中で本来何が要求されるべきかについて,我々の考えを深める契機を与えるものでもある.
もっとも,Asheim (2010)がRawls (1971)の\underline{反照的均衡}に触れながら指摘するように,
抽象的な公理を所与として,意思決定の基準がそれを満たすかどうかを議論するだけでは十分とは言えない.
具体的な文脈に照らして,
それが現実の世界にどのような帰結をもたらすのかという観点からも,
公理的分析の妥当性が問わなければならない.(阪本浩章)

\clearpage

\subsubsection*{引用参照文献}

\begin{enumerate}
  \item Asheim, G.B. (2010) ``Intergenerational equity,''
    \textit{Annual Review of Economics}, 2, 197--222.
  \item Asheim, G.B., Mitra, T., and Tungodden, B. (2012) ``Sustainable recursive social welfare functions,''
    \textit{Economic Theory}, 49(2), 267--291.
  \item Asheim, G.B., and Zuber, S. (2012) ``A complete and strongly anonymous leximin relation on infinite streams,''
    \textit{Social Choice and Welfare}, 41(4), 819--834.
  \item Basu, K. and Mitra, T. (2003) ``Aggregating infinite utility streams with intergenerational equity: the impossibility of being Paretian,''
    \textit{Econometrica}, 71(5), 1557--1563.
  \item Basu, K. and Mitra, T. (2007) ``Utilitarianism for infinite utility streams: a new welfare criterion and its axiomatic characterization,''
    \textit{Journal of Economic Theory}, 133(1), 350--373.
  \item Bossert, W, Sprumont, Y., and Suzumura, K. (2007) ``Ordering infinite utility streams,''
    \textit{Journal of Economic Theory}, 135(1), 579--589.
  \item Chichilnisky, G. (1996) ``An axiomatic approach to sustainable development,''
    \textit{Social Choice and Welfare}, 13(2), 231--257.
  \item Diamond, P. (1965) ``The evaluation of infinite utility streams,''
    \textit{Econometrica}, 33(1), 170--177.
  \item Lauwers, L. (2010) ``Ordering infinite utility streams comes at the cost of a non-Ramsey set,''
    \textit{Journal of Mathematical Economics}, 46(1), 32-37.
  \item Rawls, J. (1971) \textit{A Theory of Justice}, Harvard University Press.
  \item Sakai, T. (2010) ``Intergenerational equity and an explicit construction of welfare criteria,''
    \textit{Social Choice and Welfare}, 35(3), 393--414.
  \item Zuber, S., and Asheim, G. (2012) ``Justifying social discounting: The rank-discounted utilitarian approach,''
    \textit{Journal of Economic Theory}, 147(4), 1572--1601.
\end{enumerate}

\subsubsection*{索引語}

\begin{itemize}
  \item 公理的方法(axiomatic method)
  \item 公理(axiom)
  \item 社会的選択理論(social choice theory)
  \item 世代間衡平性(intergenerational equity)
  \item 功利主義(utilitarianism)
  \item マキシミン(maximin)
  \item レキシミン(leximin)
  \item ランク割引功利主義(rank-discounted utilitarianism)
  \item 反照的均衡(reflective equilibrium)
\end{itemize}

\end{document}